\documentclass{article}
\usepackage{xeCJK} % 支持中文
\setCJKmainfont{Songti SC} % macOS 默认支持

\usepackage{amsmath, amssymb, graphicx}
\usepackage{tcolorbox} % 美化框框
\usepackage{geometry} % 設定頁面邊距
\geometry{a4paper, margin=1in}

\begin{document}

\title{德語句子解析與語法分析}
\author{}
\date{}
\maketitle

\section{句子原文}

\begin{quote}
Als parlamentarisches Regierungssystem wird eine Staatsform, ein Organisationsmodell bezeichnet, bei dem das Volk durch gewählte Repräsentanten Staatsgewalt ausübt und das Verhältnis Parlament/Regierung nach bestimmten Prinzipien ausgeformt ist. Es geht vor allem in Konkretisierung des Demokratieprinzips um die verfassungs-
\end{quote}

\section{句子結構與語法分析}

\subsection{第一部分:Als parlamentarisches Regierungssystem wird eine Staatsform, ein Organisationsmodell bezeichnet}

\subsubsection{1️⃣ 句型結構}
\begin{itemize}
    \item \textbf{句子類型}:被動語態(Passivsatz)
    \item \textbf{動詞}:
    \begin{tcolorbox}[colframe=blue!50, colback=blue!5, sharp corners]
        \textbf{wird … bezeichnet} = 被稱為(現在時 Pr\"asens,被動語態 Passiv)
    \end{tcolorbox}
    這裡使用助動詞 \textbf{werden} + 過去分詞 \textbf{bezeichnet}。
    \item \textbf{主語(Nominativ)}:
    \begin{tcolorbox}[colframe=red!50, colback=red!5, sharp corners]
        eine Staatsform, ein Organisationsmodell(單數陰性/單數中性,均為第一格)
    \end{tcolorbox}
    \item \textbf{狀語(Adverbiale Bestimmung)}:
    \begin{tcolorbox}[colframe=green!50, colback=green!5, sharp corners]
        Als parlamentarisches Regierungssystem(作為議會制政府體系)
    \end{tcolorbox}
\end{itemize}

\subsubsection{2️⃣ 格與詞形變化}
\begin{itemize}
    \item \textbf{parlamentarisches Regierungssystem}:
    \begin{itemize}
        \item 形容詞變化:
        \begin{tcolorbox}[colframe=purple!50, colback=purple!5, sharp corners]
            \textbf{parlamentarisch} 這裡是中性弱變化(schwache Deklination)
        \end{tcolorbox}
        \item 變化規則:
        \begin{tcolorbox}[colframe=orange!50, colback=orange!5, sharp corners]
            parlamentarisch + \textbf{-es}(中性單數弱變化)
        \end{tcolorbox}
    \end{itemize}
    \item \textbf{eine Staatsform, ein Organisationsmodell}:
    \begin{itemize}
        \item 這兩個名詞都是第一格(Nominativ),因為它們是被稱為的主語。
        \item eine Staatsform(Staatsform:陰性,第一格,\textbf{eine})
        \item ein Organisationsmodell(Organisationsmodell:中性,第一格,\textbf{ein})
    \end{itemize}
\end{itemize}

\textbf{翻譯}:作為\textbf{議會制政府體系},\textbf{一種國家形式、一種組織模式}被稱為……。

\subsection{第二部分:bei dem das Volk durch gewählte Repräsentanten Staatsgewalt ausübt}
\subsubsection{1️⃣ 句型結構}
\begin{itemize}
    \item \textbf{關係從句(Relativsatz)}:
    \begin{tcolorbox}[colframe=blue!50, colback=blue!5, sharp corners]
        bei dem das Volk … ausübt
    \end{tcolorbox}
    \item \textbf{關係代詞}:
    \begin{tcolorbox}[colframe=red!50, colback=red!5, sharp corners]
        bei dem(指代 Organisationsmodell,中性,Dativ)
    \end{tcolorbox}
    \item \textbf{主語(Nominativ)}:
    \begin{tcolorbox}[colframe=green!50, colback=green!5, sharp corners]
        das Volk
    \end{tcolorbox}
    \item \textbf{動詞(Prädikat)}:
    \begin{tcolorbox}[colframe=orange!50, colback=orange!5, sharp corners]
        übt … aus(動詞「ausüben」,分離動詞,意為行使)
    \end{tcolorbox}
    \item \textbf{賓語(Akkusativ)}:
    \begin{tcolorbox}[colframe=purple!50, colback=purple!5, sharp corners]
        Staatsgewalt(第四格)
    \end{tcolorbox}
\end{itemize}

\subsubsection{2️⃣ 格與詞形變化}
\begin{itemize}
    \item \textbf{durch gewählte Repräsentanten}:
    \begin{itemize}
        \item 介詞:
        \begin{tcolorbox}[colframe=red!50, colback=red!5, sharp corners]
            durch(支配第四格,Akkusativ)
        \end{tcolorbox}
        \item 名詞:
        \begin{tcolorbox}[colframe=blue!50, colback=blue!5, sharp corners]
            Repräsentanten(複數,第四格)
        \end{tcolorbox}
        \item 形容詞變化:
        \begin{tcolorbox}[colframe=green!50, colback=green!5, sharp corners]
            gewählte(弱變化,複數第四格:gewählt + -e)
        \end{tcolorbox}
    \end{itemize}
\end{itemize}

\textbf{翻譯}:\textbf{在其中,人民透過選舉產生的代表行使國家權力。}

\subsection{總結}
這段文字使用了:
\begin{itemize}
    \item \textbf{被動語態(Passiv)}
    \item \textbf{關係從句(Relativsatz)}
    \item \textbf{介詞短語(Präpositionalphrase)}
    \item \textbf{名詞格變化(Nominativ、Akkusativ、Dativ、Genitiv)}
\end{itemize}

\end{document}